\documentclass{article}
\title{EViews' Git add-in}
\author{James Otterson}
\date{}

\usepackage{fullpage}
\usepackage{ url}

\begin{document}
\maketitle

\renewcommand{\abstractname}{Overview}
\begin{abstract}
    GitEViews is an EViews add-in that runs git commands and display their output in EViews.  It includes a 
    Git GUI entry in the Add-ins menu that starts the GUI.
\end{abstract}

\section{Summary}

Git is a version control system that can be run from a command line or from several available GUIs, including Git GUI which is packaged with Git.  This add-in 
provides a direct connection to the git command line from EViews and access to Git GUI from the Add-Ins drop-down menu.  GitEViews enables users to run 
an exact copy of a git command directly from EViews; it also displays git log output in an EViews log window.  For example, running
 
\begin{verbatim}
    git commit -m "enter message"
\end{verbatim}

\noindent from an EViews program or the EViews Command window will commit staged files with the given message.  

\subsection{GitEViews Options}
  
  The add-in has three options: 
  \begin{table}[h!]
  \centering
  \begin{tabular}{lll} \hline \hline
    Option          & Default       & Description \\ \hline
    repo\_path      & @datapath     & points to the repository location. \\ 
    log             & true          & display git logs in the EViews log window. \\
    logOptions      & -all logmsg   & only disply git log message in the log window. \\ \hline \hline \\
  \end{tabular}
  \end{table}

For example, running
\begin{verbatim}
    git(repo_path = U:/filename, log = false) commit -m "enter message"
\end{verbatim}

will commit the files in the repo located in the U:/filename folder, and it will suppress the log output. Running the code 
\begin{verbatim}
    git(repo_path = U:/filename, logOptions = all) commit -m "enter message"
\end{verbatim}
will commit the files in the repo located in the U:/filename folder, and it will display in the log window not only the 
git messages but also all other EViews messages.
\\
\\
\emph{NOTE:} To avoid the popup of a temporary workfile used to store the git log message, create a workfile called gitlogwf, like in the line below, and leave it open during your run.  
\begin{verbatim}
    wfcreate(wf=gitlogwf) u 1
\end{verbatim}    


\newpage 

\section{References on Git}

 
 \emph{The main reference on git is:}
 \\ \indent  \url{https://git-scm.com/}  
 \\ \emph{Git Command Line References:}
 \\ \indent  \url{https://git-scm.com/docs}  (reference manual)
 \\ \indent  \url{https://git-scm.com/book/en/v2}  (advanced book)
 \\ \indent  \small{\url{https://github.github.com/training-kit/downloads/github-git-cheat-sheet.pdf}} (Cheat Sheet)
 \\\emph{Reference for Git GUI:}
 \\ \indent \small{ \url{https://git-scm.com/book/en/v2/Appendix-A%3A-Git-in-Other-Environments-Graphical-Interfaces}}

\end{document}